%
\documentclass[12pt]{article}
% geometry
\usepackage[left=1in,top=1in,right=1in,bottom=1in,letterpaper]{geometry}
\usepackage{listings}
%\usepackage{algorithm,algorithmic}
\usepackage{amssymb,amsmath}
\usepackage{dsfont}
\usepackage{url}
\usepackage[lined,ruled,linesnumbered]{algorithm2e}
\lstset{language=Python}          % Set your language (you can change the language for each code-block optionally)

\usepackage[usenames]{color}
\usepackage[color,final]{showkeys}
\definecolor{refkey}{gray}{0.8}
\definecolor{labelkey}{gray}{0.8}

\usepackage[normalem]{ulem}
\newcommand{\remove}[1]{{\color{Gray}\sout{#1}}}
\newcommand{\comm}[1]{{\color{red}(#1)}}
\newcommand{\revise}[1]{{\color{blue}#1}}


\usepackage[utf8]{inputenc}

% Default fixed font does not support bold face
\DeclareFixedFont{\ttb}{T1}{txtt}{bx}{n}{12} % for bold
\DeclareFixedFont{\ttm}{T1}{txtt}{m}{n}{12}  % for normal

% Custom colors
\usepackage{color}
\definecolor{deepblue}{rgb}{0,0,0.5}
\definecolor{deepred}{rgb}{0.6,0,0}
\definecolor{deepgreen}{rgb}{0,0.5,0}


\newtheorem{theorem}{Theorem}[section]
\newtheorem{corollary}{Corollary}[theorem]
\newtheorem{lemma}[theorem]{Lemma}


\newcommand{\xx}{{\bf x}}  
\newcommand{\bb}{{\bf b}} 
\newcommand{\ba}{{\bf a}} 
\newcommand{\bc}{{\bf c}} 
\newcommand{\ee}{{\bf e}} 
\newcommand{\ff}{{\bf f}} 
\newcommand{\nn}{{\bf n}} 
\newcommand{\rr}{{\bf r}} 
\newcommand{\uu}{{\bf u}} 
\newcommand{\vv}{{\bf v}} 
\newcommand{\FF}{{\bf F}} 
\newcommand{\bA}{{\bf A}} 
\newcommand{\bB}{{\bf B}} 
\newcommand{\bC}{{\bf C}} 
\newcommand{\bE}{{\bf E}} 
\newcommand{\bS}{{\bf S}} 
\newcommand{\bX}{{\bf X}} 
\newcommand{\bT}{{\bf T}} 


\newcommand{\bnabla}{{\bf \nabla}} 
\newcommand{\btheta}{{\mathbf \theta}} 

\newcommand{\kk}{{\underline{\underline{k}}}} 
\newcommand{\cc}{{\underline{\underline c}}} 
\newcommand{\ssigma}{{\underline{\underline \sigma}}} 
\newcommand{\te}{{\underline{\underline e}}} 
\newcommand{\ta}{{\underline{\underline a}}} 

\newcommand{\RR}{{\mathbb R}}  
\newcommand{\EE}{{\mathbb E}}  

\newcommand{\GG}{{\mathcal G}}  
\newcommand{\LL}{{\mathcal L}} 
\newcommand{\UU}{{\mathcal U}} 
\newcommand{\cF}{{\mathcal F}} 

\newcommand{\dV}{{\text{d}V}} 
\newcommand{\dS}{{\text{d}S}} 

\newcommand{\Var}{\text{Var}}  
\newcommand{\std}{\text{std}}  

\newcommand{\pr}{{\partial}}  


\newcommand{\lambdat}{{\bf \lambda}}
\newcommand{\mx}[1]{\begin{pmatrix}#1\end{pmatrix}}
\newcommand{\dx}[1]{\text{ d}#1}

 
\begin{document}

\title{Math 272A Homework 3 Spring 2017}
\author{Chuyuan Fu}
\date{}

\maketitle

\section{Exercise 1}
\subsection{}
\subsubsection{plane strain}
We assume plane strain, so $\bnabla\uu$ is
\[
\bnabla \uu = \mx{\frac{\pr u}{\pr x} &\frac{\pr v}{\pr x} & 0\\
  \frac{\pr u}{\pr y} &\frac{\pr v}{\pr y} & 0\\
  0 & 0 &0
}
\]
Recall Cauchy's equation:
\[
\sigma_{xx,x} + \sigma_{xy,y} = -F_x
\]
\[
\sigma_{yx,x} + \sigma_{yy,y} = -F_y
\]
\[
\sigma_{zx} = \sigma_{zy} = 0, \sigma_{zz,z} = 0 \text{ is automatically satisfied}
\]
Substitute in
\[
\ssigma = \lambda \bnabla\cdot\uu {\mathds 1} + G(\nabla \uu +(\nabla \uu)^T), 
\]
we get the Navier's equation: 
\[
(2G + \lambda)u_{,xx} + Gu_{,yy} + (G+\lambda)v_{,xy} = -F_x
\]
\[
(2G + \lambda)v_{,xx} + Gv_{,yy} + (G+\lambda )u_{,xy} = -F_y
\]
\[
w=0
\]
In 2D,  $\uu = (u,v)$, we get the Navier's equation in 2D:
\[
(\lambda + G)\nabla(\nabla\cdot\uu) + G \nabla^2\uu=0.
\]


\subsubsection{plane stress}
In plane stress,
\[
w^{(1)} = -\frac{\lambda}{\lambda + 2G}(u_{,x}^{(0)} + v_{,y}^{(0)})
\]
Substitute into Cauchy's equation, we get
\[
(2G + \lambda - \frac{\lambda^2}{2G+\lambda})u_{,xx} + Gu_{,yy} + (G+\lambda - \frac{\lambda^2}{2G+\lambda})v_{,xy} = -F_x
\]
\[
(2G + \lambda- \frac{\lambda^2}{2G+\lambda})v_{,xx} + Gv_{,yy} + (G+\lambda- \frac{\lambda^2}{2G+\lambda})u_{,xy} = -F_y
\]
In 2D,  $\uu = (u,v)$, we get the Navier's equation in 2D:
\[
(\lambda + G- \frac{\lambda^2}{2G+\lambda})\nabla(\nabla\cdot\uu) + G \nabla^2\uu=0.
\]


\subsection{}
Assume that $u_{\theta} = rf(\theta)$, $u_r = h(r)$,
\begin{align*}
  \nabla \cdot\uu &= \frac 1 r \frac{\pr}{\pr r} (ru_r) + \frac 1 r \frac{\pr}{\pr \theta}u_{\theta} \\
  &=\frac h r + h' + f'
\end{align*}

\begin{align*}
  \bnabla (\nabla \cdot\uu) &= \frac{\pr}{\pr r} (\nabla\cdot\uu)\hat\rr + \frac 1 r \frac{\pr}{\pr \theta}(\nabla\cdot\uu)\hat{\btheta} \\
  &=(h'' + \frac{h'}{r} - \frac{h}{r^2})\hat\rr + \frac 1 r f''\hat\btheta
\end{align*}
\[
  \nabla^2 \uu = (\nabla^2u_r - \frac{u_r}{r^2} - \frac{2}{r^2} \frac{\pr u_{\theta}}{\pr {\theta}}) \hat \rr + (\nabla^2 u_{\theta} - \frac{u_{\theta}}{r^2} +
  \frac{2}{r^2}\frac{\pr u_r}{\pr \theta})\hat \theta
\]
\begin{align*}
  \nabla^2u_r &= \frac 1 r \frac{\pr}{\pr r} (r\frac{\pr u_r}{\pr r}) + \frac{1}{r^2}\frac{\pr^2 u_r}{\pr \theta^2}\\
  &=\frac 1 r \frac{\pr}{\pr r} (rh')\\
  &=\frac 1 r h' + h''
\end{align*}
\begin{align*}
  \nabla^2u_{\theta} &= \frac 1 r \frac{\pr}{\pr r} (r\frac{\pr u_r}{\pr r}) + \frac{1}{r^2}\frac{\pr^2 u_r}{\pr \theta^2}\\
  &= \frac 1 r (h' + rh'')\\
  &= \frac 1 r h' + h''
\end{align*}
Substitute the above expressions into 2D Navier's equation, we get
\begin{equation}
  (\lambda+ 2G)(h'' + \frac{h'}{r} - \frac{h}{r^2}) = \frac{2G}{r}f'
\end{equation}
\begin{equation}
  f''=0
\end{equation}




\section{Exercise 2}
\subsection{}
Let's first consider the boundary conditions for the probolem. At $x=a$, no external traction exists, so
\[
\sigma_{rr} = \sigma _{r\theta}=0.
\]
As $r\to\infty$, Mohr's circles gives that
\begin{align*}
 & \sigma_{rr} = \frac \sigma 2(1+\cos2\theta)\\
 & \sigma_{\theta\theta} = \frac \sigma 2(1-\cos2\theta)\\
 & \sigma_{r\theta} = \frac \sigma 2\sin2\theta
\end{align*}
















\section{Exercise 3}
From lecture, we know that the Airy's stress function is
\[
\Phi = Frf(\theta),
\]
where
\[
f(\theta) = a\cos\theta + b\sin\theta + c\theta\cos\theta + d\theta\sin\theta.
\]
WLOG we can assume that $a = b = 0$ (since $r\cos\theta = x$, $r\sin\theta = y$ does not contribute to stress).
\[
\sigma_{\theta\theta} = \frac{\pr^2 \Phi}{\pr r^2} =0
\]
\[
\sigma_{r\theta} = \frac {1}{ r^2} \frac{\pr \Phi}{\pr \theta} -\frac{1}{r} \frac{\pr^2 \Phi}{\pr \theta\pr r} = \frac{F}{r}(f' - f')=0
\]
\[
\sigma_{rr} = \frac 1 r \frac{\pr \Phi}{\pr r} +\frac{1}{r^2} \frac{\pr^2 \Phi}{\pr \theta^2} =\frac{F}{r}(f+f'')  = \frac F r(2c\sin\theta + 2d\cos\theta)
\]
By zero contraction on the boundary,
\[
F\ee_y +\int\sigma_{rr}\ee_r \dx{S} = 0
\]
\[
F\ee_y +\frac F r\int_{-\frac \pi 2}^{\frac \pi 2}(-2c\sin\theta + 2d\cos\theta)\mx{\cos\theta \\ \sin\theta} r\dx{S} = 0
\]
This integration gives that
\[
d=0, c = \frac 1 \pi.
\]
The Airy stress function is
\[
\Phi = Fr\frac 1 \pi \theta\cos\theta.
\]























































\end{document}


