%
\documentclass[12pt]{article}
% geometry
\usepackage[left=1in,top=1in,right=1in,bottom=1in,letterpaper]{geometry}
\usepackage{listings}
%\usepackage{algorithm,algorithmic}
\usepackage{amssymb,amsmath}
\usepackage{url}
\usepackage[lined,ruled,linesnumbered]{algorithm2e}
\lstset{language=Python}          % Set your language (you can change the language for each code-block optionally)

\usepackage[usenames]{color}
\usepackage[color,final]{showkeys}
\definecolor{refkey}{gray}{0.8}
\definecolor{labelkey}{gray}{0.8}

\usepackage[normalem]{ulem}
\newcommand{\remove}[1]{{\color{Gray}\sout{#1}}}
\newcommand{\comm}[1]{{\color{red}(#1)}}
\newcommand{\revise}[1]{{\color{blue}#1}}


\usepackage[utf8]{inputenc}

% Default fixed font does not support bold face
\DeclareFixedFont{\ttb}{T1}{txtt}{bx}{n}{12} % for bold
\DeclareFixedFont{\ttm}{T1}{txtt}{m}{n}{12}  % for normal

% Custom colors
\usepackage{color}
\definecolor{deepblue}{rgb}{0,0,0.5}
\definecolor{deepred}{rgb}{0.6,0,0}
\definecolor{deepgreen}{rgb}{0,0.5,0}


\newcommand{\xx}{{\bf x}}  
\newcommand{\bb}{{\bf b}} 
\newcommand{\cc}{{\mathbb C}} 
\newcommand{\uu}{{\bf u}} 
\newcommand{\FF}{{\bf F}} 


\newcommand{\RR}{{\mathbb R}}  
\newcommand{\EE}{{\mathbb E}}  

\newcommand{\Var}{\text{Var}}  
\newcommand{\std}{\text{std}}  

\newcommand{\lambdat}{{\bf \lambda}}
\newcommand{\mx}[1]{\begin{pmatrix}#1\end{pmatrix}}
\newcommand{\dx}[1]{\text{ d\,}#1}


\begin{document}

\title{Math 272A Spring 2017}
\author{Chuyuan Fu}
\date{}

\maketitle

\section{What is a continuum?}

The quantities that we are interested in, for example displacement, velocity, pressure, stress... can be written as fields.
For example, we can define a displacement field $\uu(\xx, t) : \RR^3\times \RR \rightarrow \RR^3$, which tells us how the matter in a body is displaced when forces are applied. At each point $(\xx,t)$, $\uu$ gives the average displacement of all molecules contained in a small box that is centered at $\xx$.

Specifically, we assume the existence of a length scale $l$ such that if we average over the displacements of all atoms in a box of size $l$, we get a well-defined average.

Imagine we are interested in a quantity $f$. Each molecule has its own value of $f_i$. Lets assume that the $f_i$'s are fluactuating but have the same mean
\[
f_i = \bar f + f_i'
\]
where $\EE f_i'$=0.
Assume Poisson fluctuations $\std(f_i') = \bar f$.
Then our estimator of $\bar f $ is 
\[
\hat f  = \frac 1 N \sum_i f_i
\]
\begin{align*}
\Var (\hat f)  &=  \Var(\sum_i  \frac 1 N f_i' )\\
		&= \frac {1}{ N^2} \sum_i \Var f_i' \\
		&= \frac {1}{ N^2} N \bar f =\frac{\bar f}{N}
\end{align*}
\[
\std(\hat f ) = \frac{\bar f }{\sqrt{N}}.
\]
Relative error is
\[ 
\frac{\std(\hat f )}{\bar f} = \frac{1}{\sqrt N}
\]
Practically, if I want 0.001 relative error, $N = 10^6$, so I want 100 molecules per side of the box, i.e. need $l \approx 100\times d  = 10$ nm, where d is a molecule diameter, $d = 10^{-10}$m. 
Many of the objects we want to study are larger than this, e.g. in a microfluid device, channel diameter is $10\times 10^{-6} $ m, cell diameter is $10^{-6}$ - $10^{-5}$ m, cell membrane thickness is 10 - 30 nm, carbon nanotube diameter 2 - 200 nm.

In all of the calculus arguments that follow, we will be taking limits, e.g taking a small volume, $\delta V$ and letting $\delta V \rightarrow 0$. In fact, we assume  $\delta V \rightarrow 0$ but $\delta V > l^3$.

\section{Principles of Linear Elasticity}
Assume we have a body to which a discrete set of estimated forces $\{\FF_i\}$ is applied under the constraints
\begin{align*}
&\sum_i \FF_i = 0 \\
&\sum_i \xx_i\times \FF_i = 0 \text{ ( independent of choice of origin)}
\end{align*}

Hooke's 3 laws for linear elasticity
\begin{itemize}
\item If no forces are applied to the body ($\FF_i = 0$), the body has a unique rest state.
we can measure displacements relative to this state.
\item If we are interested in the displacement $\uu$ at a particular point $\uu(P)$ ther exists a set of coeffcicient matrices $\cc_i(P)$ such that  $\uu(P) = \sum_i \cc_i(P) \cdot \FF_i$
\end{itemize}


Because of Hooke's 2nd hypothesis, displacements are linear in forces in the sense that 
\begin{itemize}
\item if $\{\FF_i\}$ is a set of forces and $\uu(P)$ is the corresponding displacement, then: 
If we instead consider the set of forces $\{\lambda \FF_i\}$ then the displacement will be $\lambda \uu(P)$

\item If $\FF_i$ is a force and $\uu(P)$ is the corresponding displacement, $\FF_i'$ another force and $\uu'(P)$ its corresponding displacement, then if we apply $\FF_i + \FF_i'$ the displacement will be $\uu(P) +\uu'(P)$ 
\end{itemize}

Consider the case where the direction of the force at $i$ is set in advance; then the force at $i$ is given by a scalar $F_i$. If w want to have the freedom to change the direction of the forces, then we can create 3 pts $i$, $i+1$, $i+2$ that all lie in the same point in the body and in which the directions of $F_i$, $F_{i+1}$, $F_{i+2}$ are a 3D basis. Suppose also that I am interested in the displacements only the points indexed by $i$, and only in this direction of the forces at $i$

then Hooke's first and second hypothesis tells us that the displacements $\{u_i\}$
and forces $F_i$ are related by
\[
u_i
\sum_i c_{ij}F_j
\]
for some set of coeefcients $c_{ij} $ or equiv
\[
\uu = \cc\cdot\FF
\]
where $\uu = (u_1, u_2 ... u_N)$, $\cc$ is a $N\times N$ matrix of coefficients, $c_{ij}$ are called the influence coefficients.

We would like to calculate the work done by the forces $\{F_i\}$ in deforming the body. Recall if a force $\FF$ acts on a body and moves the body a distance $\uu$ then the work done is 
\[
\FF\cdot\uu
\]
we want to calculate the total work done by the forces $\{F_i\}$ in creating displacements $\{u_i\}$ is NOT $\sum_i F_i u_i$.

Imagine that I build up the forces $\{F_i\}$ starting with all $F_i=0$ and keeping all proportions $F_i/F_j$ constant.
If $\{F_i^*\}$ is the target force, I build up $F_i(t)$ over time by 
\[
F_i(t) = F_i^*t.
\]
Similarly if the end displacements are $u_i^*$ then $u_i(t)$ builds up over time by 
\[
u_i(t) = u_i^*t = \sum_j c_{ij} F_j^*t
\]
When displacements increase from $u_i$ to $u_i+\delta u_i$, where $\delta u_i= \sum_j c_{ij} F_j^*\delta t$, work done is 
\begin{align*}
\sum_i F_i \delta u_i &= \sum_{i,j} F_ic_{ij}F_j^*\delta t\\
	&= \sum_{i,j} F_i^*tc_{ij}F_j^*\delta t\\
	&= \sum_{i,j} c_{ij}F_i^*F_j^*t\delta t
\end{align*}

Total work done is 
\begin{align*}
U &= \sum_{i,j} c_{ij}F_i^*F_j^*t\int_0^1 t \dx{t}\\
	&=\frac 1 2 \sum_{i,j} c_{ij}F_i^*F_j^*\\
	&=\frac 1 2 \sum_i F_i^* u_i^*
\end{align*}

Hypothesis 3
The total work done in deforming the body should be 
\begin{itemize}
\item Positive unless $F_i=0, \forall i$.
\item Independent of the way in which the forces build up from 0 to $\{F_i\}$
\end{itemize}

The first part implies that $\cc$ is a positive definite matrix.
The second part implies the following:
I want $U(\FF^*)$ to be independent of the path going from 0 to $\FF^*$. Work done in traversing the path is
\begin{align*}
U = \int_{path} \sum_iF_i\delta u_i\\
	=  \int_{path} \sum_{i,j} F_i c_{ij}\delta F_j\
\end{align*} 
i.e. I need the above to be an exact differential. An exact differential is a quantity of the form 
\[
\sum\frac{\partial f}{\partial x_i} \dx{x_i}
\]
A quantity $\sum_i p_i(\xx) \dx{x_i}$ is an exact differential if and only if 
\[
\frac{\partial p_i}{\partial x_j} = \frac{\partial p_j}{\partial x_i}.
\]



















\end{document}


