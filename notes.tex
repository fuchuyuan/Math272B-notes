%
\documentclass[12pt]{article}
% geometry
\usepackage[left=1in,top=1in,right=1in,bottom=1in,letterpaper]{geometry}
\usepackage{listings}
%\usepackage{algorithm,algorithmic}
\usepackage{amssymb,amsmath}
\usepackage{url}
\usepackage[lined,ruled,linesnumbered]{algorithm2e}
\lstset{language=Python}          % Set your language (you can change the language for each code-block optionally)

\usepackage[usenames]{color}
\usepackage[color,final]{showkeys}
\definecolor{refkey}{gray}{0.8}
\definecolor{labelkey}{gray}{0.8}

\usepackage[normalem]{ulem}
\newcommand{\remove}[1]{{\color{Gray}\sout{#1}}}
\newcommand{\comm}[1]{{\color{red}(#1)}}
\newcommand{\revise}[1]{{\color{blue}#1}}


\usepackage[utf8]{inputenc}

% Default fixed font does not support bold face
\DeclareFixedFont{\ttb}{T1}{txtt}{bx}{n}{12} % for bold
\DeclareFixedFont{\ttm}{T1}{txtt}{m}{n}{12}  % for normal

% Custom colors
\usepackage{color}
\definecolor{deepblue}{rgb}{0,0,0.5}
\definecolor{deepred}{rgb}{0.6,0,0}
\definecolor{deepgreen}{rgb}{0,0.5,0}


\newtheorem{theorem}{Theorem}[section]
\newtheorem{corollary}{Corollary}[theorem]
\newtheorem{lemma}[theorem]{Lemma}

\newcommand{\xx}{{\bf x}}  
\newcommand{\bb}{{\bf b}} 
\newcommand{\ee}{{\bf e}} 
\newcommand{\uu}{{\bf u}} 
\newcommand{\FF}{{\bf F}} 
\newcommand{\bA}{{\bf A}} 

\newcommand{\bnabla}{{\bf \nabla}} 

\newcommand{\kk}{{\underline{\underline{k}}}} 
\newcommand{\cc}{{\underline{\underline c}}} 

\newcommand{\RR}{{\mathbb R}}  
\newcommand{\EE}{{\mathbb E}}  

\newcommand{\GG}{{\mathcal G}}  
\newcommand{\LL}{{\mathcal L}} 
\newcommand{\UU}{{\mathcal U}} 
\newcommand{\cF}{{\mathcal F}} 

\newcommand{\Var}{\text{Var}}  
\newcommand{\std}{\text{std}}  

\newcommand{\lambdat}{{\bf \lambda}}
\newcommand{\mx}[1]{\begin{pmatrix}#1\end{pmatrix}}
\newcommand{\dx}[1]{\text{ d\,}#1}


\begin{document}

\title{Math 272A Spring 2017}
\author{Chuyuan Fu}
\date{}

\maketitle

\section{What is a continuum?}

The quantities that we are interested in, for example displacement, velocity, pressure, stress... can be written as fields.
For example, we can define a displacement field $\uu(\xx, t) : \RR^3\times \RR \rightarrow \RR^3$, which tells us how the matter in a body is displaced when forces are applied. At each point $(\xx,t)$, $\uu$ gives the average displacement of all molecules contained in a small box that is centered at $\xx$.

Specifically, we assume the existence of a length scale $l$ such that if we average over the displacements of all atoms in a box of size $l$, we get a well-defined average.

Imagine we are interested in a quantity $f$. Each molecule has its own value of $f_i$. Lets assume that the $f_i$'s are fluactuating but have the same mean
\[
f_i = \bar f + f_i'
\]
where $\EE f_i'$=0.
Assume Poisson fluctuations $\std(f_i') = \bar f$.
Then our estimator of $\bar f $ is 
\[
\hat f  = \frac 1 N \sum_i f_i
\]
\begin{align*}
\Var (\hat f)  &=  \Var(\sum_i  \frac 1 N f_i' )\\
		&= \frac {1}{ N^2} \sum_i \Var f_i' \\
		&= \frac {1}{ N^2} N \bar f =\frac{\bar f}{N}
\end{align*}
\[
\std(\hat f ) = \frac{\bar f }{\sqrt{N}}.
\]
Relative error is
\[ 
\frac{\std(\hat f )}{\bar f} = \frac{1}{\sqrt N}
\]
Practically, if I want 0.001 relative error, $N = 10^6$, so I want 100 molecules per side of the box, i.e. need $l \approx 100\times d  = 10$ nm, where d is a molecule diameter, $d = 10^{-10}$m. 
Many of the objects we want to study are larger than this, e.g. in a microfluid device, channel diameter is $10\times 10^{-6} $ m, cell diameter is $10^{-6}$ - $10^{-5}$ m, cell membrane thickness is 10 - 30 nm, carbon nanotube diameter 2 - 200 nm.

In all of the calculus arguments that follow, we will be taking limits, e.g taking a small volume, $\delta V$ and letting $\delta V \rightarrow 0$. In fact, we assume  $\delta V \rightarrow 0$ but $\delta V > l^3$.

\section{Principles of Linear Elasticity}
Assume we have a body to which a discrete set of estimated forces $\{\FF_i\}$ is applied under the constraints
\begin{align*}
&\sum_i \FF_i = 0 \\
&\sum_i \xx_i\times \FF_i = 0 \text{ ( independent of choice of origin)}
\end{align*}

Hooke's 3 laws for linear elasticity
\begin{itemize}
\item If no forces are applied to the body ($\FF_i = 0$), the body has a unique rest state.
we can measure displacements relative to this state.
\item If we are interested in the displacement $\uu$ at a particular point $\uu(P)$ ther exists a set of coeffcicient matrices $\cc_i(P)$ such that  $\uu(P) = \sum_i \cc_i(P) \cdot \FF_i$
\end{itemize}


Because of Hooke's 2nd hypothesis, displacements are linear in forces in the sense that 
\begin{itemize}
\item if $\{\FF_i\}$ is a set of forces and $\uu(P)$ is the corresponding displacement, then: 
If we instead consider the set of forces $\{\lambda \FF_i\}$ then the displacement will be $\lambda \uu(P)$

\item If $\FF_i$ is a force and $\uu(P)$ is the corresponding displacement, $\FF_i'$ another force and $\uu'(P)$ its corresponding displacement, then if we apply $\FF_i + \FF_i'$ the displacement will be $\uu(P) +\uu'(P)$ 
\end{itemize}

Consider the case where the direction of the force at $i$ is set in advance; then the force at $i$ is given by a scalar $F_i$. If w want to have the freedom to change the direction of the forces, then we can create 3 pts $i$, $i+1$, $i+2$ that all lie in the same point in the body and in which the directions of $F_i$, $F_{i+1}$, $F_{i+2}$ are a 3D basis. Suppose also that I am interested in the displacements only the points indexed by $i$, and only in this direction of the forces at $i$

then Hooke's first and second hypothesis tells us that the displacements $\{u_i\}$
and forces $F_i$ are related by
\[
u_i
\sum_i c_{ij}F_j
\]
for some set of coeefcients $c_{ij} $ or equiv
\[
\uu = \cc\cdot\FF
\]
where $\uu = (u_1, u_2 ... u_N)$, $\cc$ is a $N\times N$ matrix of coefficients, $c_{ij}$ are called the influence coefficients.

We would like to calculate the work done by the forces $\{F_i\}$ in deforming the body. Recall if a force $\FF$ acts on a body and moves the body a distance $\uu$ then the work done is 
\[
\FF\cdot\uu
\]
we want to calculate the total work done by the forces $\{F_i\}$ in creating displacements $\{u_i\}$ is NOT $\sum_i F_i u_i$.

Imagine that I build up the forces $\{F_i\}$ starting with all $F_i=0$ and keeping all proportions $F_i/F_j$ constant.
If $\{F_i^*\}$ is the target force, I build up $F_i(t)$ over time by 
\[
F_i(t) = F_i^*t.
\]
Similarly if the end displacements are $u_i^*$ then $u_i(t)$ builds up over time by 
\[
u_i(t) = u_i^*t = \sum_j c_{ij} F_j^*t
\]
When displacements increase from $u_i$ to $u_i+\delta u_i$, where $\delta u_i= \sum_j c_{ij} F_j^*\delta t$, work done is 
\begin{align*}
\sum_i F_i \delta u_i &= \sum_{i,j} F_ic_{ij}F_j^*\delta t\\
	&= \sum_{i,j} F_i^*tc_{ij}F_j^*\delta t\\
	&= \sum_{i,j} c_{ij}F_i^*F_j^*t\delta t
\end{align*}

Total work done is 
\begin{align*}
U &= \sum_{i,j} c_{ij}F_i^*F_j^*t\int_0^1 t \dx{t}\\
	&=\frac 1 2 \sum_{i,j} c_{ij}F_i^*F_j^*\\
	&=\frac 1 2 \sum_i F_i^* u_i^*
\end{align*}

Hypothesis 3
The total work done in deforming the body should be 
\begin{itemize}
\item Positive unless $F_i=0, \forall i$.
\item Independent of the way in which the forces build up from 0 to $\{F_i\}$
\end{itemize}

The first part implies that $\cc$ is a positive definite matrix.
The second part implies the following:
I want $U(\FF^*)$ to be independent of the path going from 0 to $\FF^*$. Work done in traversing the path is
\begin{align*}
U = \int_{path} \sum_iF_i\delta u_i\\
	=  \int_{path} \sum_{i,j} F_i c_{ij}\delta F_j\
\end{align*} 
i.e. I need the above to be an exact differential. An exact differential is a quantity of the form 
\[
\sum\frac{\partial f}{\partial x_i} \dx{x_i}
\]
A quantity $\sum_i p_i(\xx) \dx{x_i}$ is an exact differential if and only if 
\[
\frac{\partial p_i}{\partial x_j} = \frac{\partial p_j}{\partial x_i}.
\]
This means that 
\[
\frac{\partial F_kc_{ki}}{\partial F_j} = \frac{\partial F_kc_{kj}}{\partial F_i} 
\]
\[c_{ji} = c_{ij}\]
$\cc$ is symmetric.

This means that if we apply a point force a point $i$ and measure displacement at $j$, we get the same displacement as if we apply the force at $j$ and measure displacement at $i$.
$\cc = \cc^T$ is known as Maxwell-reciprocity. This is analogous to self-adjointness of ODEs/PDEs. e.g. if I have an ODE
\[
\LL[y] = f(x)
\]
I can construct a Green's function solution
\[
\LL_x[\GG(x,\tilde x )] = \delta(x-\tilde x )
\]
If $\LL$ is a self-adjoint operator then 
\[
\GG(x, \tilde x) = \GG(\tilde x, x).
\]
The PDEs for linear elasticity theory are self-adjoint.

\begin{enumerate}
\item
Suppose we load (i.e. apply a set of forces) to an elastic body doing an amount of work $U = \frac 1 2 \sum_{i,j}c_{ij}F_iF_j$.
Now we take the forces away; in remaining the forces, we must do work $-U$ i.e. the forces within the body do work $U$. we can think of $U$ as the {\bf internal elastic energy} within a deformed body.
 
 \item {Lorentzian or Betti-Rayleigh reciprocity.}
 Consider the same body deformed under two different set of forces $\{F_i\}, \{F_i'\}$with reps displacements $\{u_i\}, \{u_i'\}$. Index notation below
 \[
F_i'u_i = F_i' c_{ij}F_j =  F_i' c_{ji}F_j = F_i c_{ij}F_j' = F_i u_i' 
 \]
 This is known to a bunch of Harvard people (including Roper) as {\bf something for nothing formula}.
 
 \item {Uniqueness.} Using $\cc$ I can calculate $\{u_i\}$ from $\{F_i\}$ using $\uu  = \cc\cdot\FF$
 I can similarly calculate $\FF$ (uniquely) from $\uu$ as 
 \[
\FF = \kk\cdot \uu
 \]
 where $\kk= \cc^{-1}$ and $\cc$ is invertible because it is (by Hooke's first law).
 $\kk$ is called the matrix of spring constants.
 \[
 U= \frac 1 2 \sum_i F_i u_i = \frac 1 2 \sum_{ij} k_{ij}u_iu_j
 \]
 
 
 \item Principle of virtual work.
 \[
\frac{\partial U}{\partial u_i} = \sum_j k_{ij}u_j = F_i
 \]
 Take a stressed and already deformed body; and imagine applying additional virtual displacements $\{\delta u_i\}$ to the body. The external forces $F_i$ do ``virtual work": external virtual work (EVW) $ \sum_i F_i\delta u_i$.
 
 I also change the elastic (internal) energy of the body by 
 \[
\delta U = \sum_i \frac{\partial U}{\partial u_i} \delta u_i
 \]
 also known as "internal virtual work" (IVW). The idea IVW = EVW is key to Finite Element Models for linear elasticity. The principle of virtual work is often assumed and can be used to solve problems with {\bf large deformations}, membranes, etc.
\end{enumerate}

\bigskip
\bigskip
Often our b.c.s may consist of a combination of known forces and known displacements. A loaded bridge has $\uu=0$ at points of support. Is he displacement field unique when we use mixed b.c.s.?

Assume we have two set of b.c.s. $\cF$ where $F_i$ is know, $\UU$ where $u_i$ is known, $\cF$ and $\UU$ disjoint.

Can we specify a set of forces $F_i^*$ such that 
\[
F_i^*  = F_i \quad  \forall i\in \cF
\]
\[
\sum_jc_{ij}F_j^*  = u_i \quad \forall i\in \UU
\]

We need to solve $|\UU|$ equqtions to calculate $|\UU|$ unknown $F_i^*$
\[
\sum_{j\in \UU} c_{ij}F_j^* = u_i - \sum_{j\in\cF} c_{ij}F_j^*
\]

Because $\cc$ is positive deinite, all of its principal minors are positive. We can order the indices $i$ so that the first $|\UU|$ indices are associated with $\UU$ and the remaining indices with $\cF$.
Because $[\cc]\in\UU\times\UU$ has positive determinant, the above equation can be uniquely solved.

Method for finding approximate (even exact) solutions to equations.
Call a set of forces $\{F_i^*\}$ admissible if $\{F_i^*\} = F_i$ $\forall i\in\cF$.
Claim: the unique admissible solution of minimized the complementary energy
\[
V(F_i^*) = U(F_i^*) - \sum_{i\in\UU} F_i^*u_i
\]
(this is a Legendre transform.)

To minimized $V$ we need $\frac{\partial V}{\partial F_i^*}=0$ $\forall i\in\UU$:
\[
\sum_jc_{ij}F_j^* - u_i=0 \quad \forall i\in\UU.
\]
Still need to show that $\{F_i^*\}$ is a global minimizer. To do this we show that $V\to \infty$ in all directions.
By positive definiteness of $\cc$, 
\[
\sum_{ij} c_{ij}  F_i^* F_j^* \geq \lambda_{\min}\|\FF\|^2,
\]
whereas
\[
|\sum_{i\in\UU}u_iF_i^*| \leq u_{\max} \|\FF\|.
\]
So $V\to\infty$ as $\|\FF\|\to\infty$.

This is known as Principle of minimum complementary energy.

We can instead write down our problem in elasticity theory. 
We specify $u_i$, $i\in\UU$, and we want to solve for  $u_i$, $i\in\cF$. There will be unique $u_i$, $i\in\cF$ satisfying all equations; and if we define admissible displacements $u_i^*$ such that  $u_i^* = u_i$ $\forall i\in\UU$ and $u_i^*$ unconstrained $\forall i\in\cF$ and the unique solution to the problem minimizes 
\[
V(u_i^*) = U(F_i^*) - \sum_{i\in\cF} F_i^*u_i
\]

\section{Tensors}
\subsection{Cartesian tensors and vector calculus \& summation convention}
 Components of vectors are related via transformation/rotation matrices.
\[
l_{ij} = \ee_i'\cdot\ee_j.
\]
If we have a vector $\FF$ whose components are $F_i$ in the ``old" coordinate system,
\[
\FF = F_i\ee_i
\]
using summation convention. The components of $\FF$ in the new coordinate system are
\[
\FF = F_i'\ee_i',
\]
where 
\[
F_i' = \FF\cdot\ee_i'= F_j\ee_j\cdot\ee_i'
\]
\begin{equation}\label{eq:3.1}
F_i' = l_{ij} F_j.
\end{equation}
A vector in $\RR^3$ is a set of 3 quantities $\FF = \mx{F_1\\F_2\\F_3}$ that when the coordinate system is changed, transforms according to \ref{eq:3.1}. Examples include forces, velocotiy, displacements, position, torque.

we will define a rank $n$ tensor $\underline{\underset{\vdots}{\underline\bA}}$ ( underlined $n$ times) that is made up of $3^n$ components in $\RR$, $A_{i_1, i_2, \dots,  i_n}$. When we transform coordinates, these components will transform in a way that generalize \ref{eq:3.1}:
\[
A_{i_1, i_2\dotsi_n} = l_{i_1j_1} l_{i_2j_2} \cdots l_{i_nj_n}A_{j_1, j_2\dots j_n}.
\]
e.g. 

$n=0$, $A$ is s scalar. e.g., temperature, pressure, energy.

$n=1$ $\underline A$ is a vector that transforms by matrix multiply.

$n=2$ $\underline{\underline A}$ is an array of 9 numbers that transforms by 
\[
A' = LAL^T, \quad L = (l_{ij}),
\]
\[
A' = l_{ik} l_{jm} A_{km} =  l_{ik} A_{km} (l^T)_{mj} 
\]
where $A$ is an old set of 9 numbers, $A'$ is the new set of 9 numbers. e.g. include: stress, strain (or rate of strain).

\begin{theorem} (see Landau \& Lifschitz)
Define a {\bf isotropic} tensor to be a tensor $\underline{\underline A}$ whose coordinate components are independent of the {\bf allowed} choice of coordinate system.
\end{theorem}
In $\RR^3$,

$n=0$, all scalars are isotropic.

$n=1$, $\bf 0$ is the only isotropic vector.

$n=2$, the only isotropic tensors have components, 
\[
A_{ij} = \lambda \delta_{ij}.
\]

$n=3$, only isotropic tensor is the alternating tensor
\[
\lambda\epsilon_{ijk}.
\]

$n=4$, $\lambda\delta_{ij}\delta_{kl}$,  $\mu\delta_{ik}\delta_{jl}$,  $\nu\delta_{il}\delta_{jk}$ and linear combinations thereof.

\bigskip
\bigskip

{\bf Important identity}
\[
\epsilon_{ijk}\epsilon_{ilm} = \delta_{jl}\delta_{km} - \delta_{jm}\delta_{kl}.
\]

Using summation convenction to evaluation some vector calculus quantities.
\[
\bnabla(\frac{\xx}{r}), \text{ where } r = \sqrt{x^2 + y^2 + z^2}.
\]
\[
[\bnabla(\frac{\xx}{r})]_{ij}= \frac{\partial}{\partial x_i}\frac{x_j}{r} = \frac{\delta_{ij}}{r} - \frac{x_ix_j}{r^3}
\]








































\end{document}


