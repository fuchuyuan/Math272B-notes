%
\documentclass[12pt]{article}
% geometry
\usepackage[left=1in,top=1in,right=1in,bottom=1in,letterpaper]{geometry}
\usepackage{listings}
%\usepackage{algorithm,algorithmic}
\usepackage{amssymb,amsmath}
\usepackage{dsfont}
\usepackage{url}
\usepackage[lined,ruled,linesnumbered]{algorithm2e}
\lstset{language=Python}          % Set your language (you can change the language for each code-block optionally)

\usepackage[usenames]{color}
\usepackage[color,final]{showkeys}
\definecolor{refkey}{gray}{0.8}
\definecolor{labelkey}{gray}{0.8}

\usepackage[normalem]{ulem}
\newcommand{\remove}[1]{{\color{Gray}\sout{#1}}}
\newcommand{\comm}[1]{{\color{red}(#1)}}
\newcommand{\revise}[1]{{\color{blue}#1}}


\usepackage[utf8]{inputenc}

% Default fixed font does not support bold face
\DeclareFixedFont{\ttb}{T1}{txtt}{bx}{n}{12} % for bold
\DeclareFixedFont{\ttm}{T1}{txtt}{m}{n}{12}  % for normal

% Custom colors
\usepackage{color}
\definecolor{deepblue}{rgb}{0,0,0.5}
\definecolor{deepred}{rgb}{0.6,0,0}
\definecolor{deepgreen}{rgb}{0,0.5,0}


\newtheorem{theorem}{Theorem}[section]
\newtheorem{corollary}{Corollary}[theorem]
\newtheorem{lemma}[theorem]{Lemma}


\newcommand{\xx}{{\bf x}}  
\newcommand{\bb}{{\bf b}} 
\newcommand{\ee}{{\bf e}} 
\newcommand{\ff}{{\bf f}} 
\newcommand{\nn}{{\bf n}} 
\newcommand{\rr}{{\bf r}} 
\newcommand{\uu}{{\bf u}} 
\newcommand{\vv}{{\bf v}} 
\newcommand{\FF}{{\bf F}} 
\newcommand{\bA}{{\bf A}} 
\newcommand{\bB}{{\bf B}} 
\newcommand{\bC}{{\bf C}} 
\newcommand{\bE}{{\bf E}} 
\newcommand{\bS}{{\bf S}} 
\newcommand{\bX}{{\bf X}} 
\newcommand{\bT}{{\bf T}} 

\newcommand{\bnabla}{{\bf \nabla}} 

\newcommand{\kk}{{\underline{\underline{k}}}} 
\newcommand{\cc}{{\underline{\underline c}}} 
\newcommand{\ssigma}{{\underline{\underline \sigma}}} 
\newcommand{\te}{{\underline{\underline e}}} 
\newcommand{\ta}{{\underline{\underline a}}} 

\newcommand{\RR}{{\mathbb R}}  
\newcommand{\EE}{{\mathbb E}}  

\newcommand{\GG}{{\mathcal G}}  
\newcommand{\LL}{{\mathcal L}} 
\newcommand{\UU}{{\mathcal U}} 
\newcommand{\cF}{{\mathcal F}} 

\newcommand{\dV}{{\text{d}V}} 
\newcommand{\dS}{{\text{d}S}} 

\newcommand{\Var}{\text{Var}}  
\newcommand{\std}{\text{std}}  

\newcommand{\lambdat}{{\bf \lambda}}
\newcommand{\mx}[1]{\begin{pmatrix}#1\end{pmatrix}}
\newcommand{\dx}[1]{\text{ d}#1}


\begin{document}

\title{Math 272A Spring 2017}
\author{Chuyuan Fu}
\date{}

\maketitle
\section{What is a continuum?}
The quantities that we are interested in, for example displacement, velocity, pressure, stress... can be written as fields.
For example, we can define a displacement field $\uu(\xx, t) : \RR^3\times \RR \rightarrow \RR^3$, which tells us how the matter in a body is displaced when forces are applied. At each point $(\xx,t)$, $\uu$ gives the average displacement of all molecules contained in a small box that is centered at $\xx$.

Specifically, we assume the existence of a length scale $l$ such that if we average over the displacements of all atoms in a box of size $l$, we get a well-defined average.

Imagine we are interested in a quantity $f$. Each molecule has its own value of $f_i$. Lets assume that the $f_i$'s are fluctuating but have the same mean
\[
f_i = \bar f + f_i'
\]
where $\EE f_i'$=0.
Assume Poisson fluctuations $\std(f_i') = \bar f$.
Then our estimator of $\bar f $ is 
\[
\hat f  = \frac 1 N \sum_i f_i
\]
\begin{align*}
\Var (\hat f)  &=  \Var(\sum_i  \frac 1 N f_i' )\\
		&= \frac {1}{ N^2} \sum_i \Var f_i' \\
		&= \frac {1}{ N^2} N \bar f =\frac{\bar f}{N}
\end{align*}
\[
\std(\hat f ) = \frac{\bar f }{\sqrt{N}}.
\]
Relative error is
\[ 
\frac{\std(\hat f )}{\bar f} = \frac{1}{\sqrt N}
\]
Practically, if I want 0.001 relative error, $N = 10^6$, so I want 100 molecules per side of the box, i.e. need $l \approx 100\times d  = 10$ nm, where d is a molecule diameter, $d = 10^{-10}$m. 
Many of the objects we want to study are larger than this, e.g. in a microfluid device, channel diameter is $10\times 10^{-6} $ m, cell diameter is $10^{-6}$ - $10^{-5}$ m, cell membrane thickness is 10 - 30 nm, carbon nanotube diameter 2 - 200 nm.

In all of the calculus arguments that follow, we will be taking limits, e.g taking a small volume, $\delta V$ and letting $\delta V \rightarrow 0$. In fact, we assume  $\delta V \rightarrow 0$ but $\delta V > l^3$.

\section{Principles of Linear Elasticity}
Assume we have a body to which a discrete set of estimated forces $\{\FF_i\}$ is applied under the constraints
\begin{align*}
&\sum_i \FF_i = 0 \\
&\sum_i \xx_i\times \FF_i = 0 \text{ ( independent of choice of origin)}
\end{align*}

Hooke's 3 laws for linear elasticity
\begin{itemize}
\item If no forces are applied to the body ($\FF_i = 0$), the body has a unique rest state.
we can measure displacements relative to this state.
\item If we are interested in the displacement $\uu$ at a particular point $\uu(P)$ there exists a set of coefficient matrices $\cc_i(P)$ such that  $\uu(P) = \sum_i \cc_i(P) \cdot \FF_i$
\end{itemize}


Because of Hooke's 2nd hypothesis, displacements are linear in forces in the sense that 
\begin{itemize}
\item if $\{\FF_i\}$ is a set of forces and $\uu(P)$ is the corresponding displacement, then: 
If we instead consider the set of forces $\{\lambda \FF_i\}$ then the displacement will be $\lambda \uu(P)$

\item If $\FF_i$ is a force and $\uu(P)$ is the corresponding displacement, $\FF_i'$ another force and $\uu'(P)$ its corresponding displacement, then if we apply $\FF_i + \FF_i'$ the displacement will be $\uu(P) +\uu'(P)$ 
\end{itemize}

Consider the case where the direction of the force at $i$ is set in advance; then the force at $i$ is given by a scalar $F_i$. If w want to have the freedom to change the direction of the forces, then we can create 3 pts $i$, $i+1$, $i+2$ that all lie in the same point in the body and in which the directions of $F_i$, $F_{i+1}$, $F_{i+2}$ are a 3D basis. Suppose also that I am interested in the displacements only the points indexed by $i$, and only in this direction of the forces at $i$

then Hooke's first and second hypothesis tells us that the displacements $\{u_i\}$
and forces $F_i$ are related by
\[
u_i
\sum_i c_{ij}F_j
\]
for some set of coefficients $c_{ij} $ or equivalently
\[
\uu = \cc\cdot\FF
\]
where $\uu = (u_1, u_2 ... u_N)$, $\cc$ is a $N\times N$ matrix of coefficients, $c_{ij}$ are called the influence coefficients.

We would like to calculate the work done by the forces $\{F_i\}$ in deforming the body. Recall if a force $\FF$ acts on a body and moves the body a distance $\uu$ then the work done is 
\[
\FF\cdot\uu
\]
we want to calculate the total work done by the forces $\{F_i\}$ in creating displacements $\{u_i\}$ is NOT $\sum_i F_i u_i$.

Imagine that I build up the forces $\{F_i\}$ starting with all $F_i=0$ and keeping all proportions $F_i/F_j$ constant.
If $\{F_i^*\}$ is the target force, I build up $F_i(t)$ over time by 
\[
F_i(t) = F_i^*t.
\]
Similarly if the end displacements are $u_i^*$ then $u_i(t)$ builds up over time by 
\[
u_i(t) = u_i^*t = \sum_j c_{ij} F_j^*t
\]
When displacements increase from $u_i$ to $u_i+\delta u_i$, where $\delta u_i= \sum_j c_{ij} F_j^*\delta t$, work done is 
\begin{align*}
\sum_i F_i \delta u_i &= \sum_{i,j} F_ic_{ij}F_j^*\delta t\\
	&= \sum_{i,j} F_i^*tc_{ij}F_j^*\delta t\\
	&= \sum_{i,j} c_{ij}F_i^*F_j^*t\delta t
\end{align*}

Total work done is 
\begin{align*}
U &= \sum_{i,j} c_{ij}F_i^*F_j^*t\int_0^1 t \dx{t}\\
	&=\frac 1 2 \sum_{i,j} c_{ij}F_i^*F_j^*\\
	&=\frac 1 2 \sum_i F_i^* u_i^*
\end{align*}

Hypothesis 3
The total work done in deforming the body should be 
\begin{itemize}
\item Positive unless $F_i=0, \forall i$.
\item Independent of the way in which the forces build up from 0 to $\{F_i\}$
\end{itemize}

The first part implies that $\cc$ is a positive definite matrix.
The second part implies the following:
I want $U(\FF^*)$ to be independent of the path going from 0 to $\FF^*$. Work done in traversing the path is
\begin{align*}
U = \int_{path} \sum_iF_i\delta u_i\\
	=  \int_{path} \sum_{i,j} F_i c_{ij}\delta F_j\
\end{align*} 
i.e. I need the above to be an exact differential. An exact differential is a quantity of the form 
\[
\sum\frac{\partial f}{\partial x_i} \dx{x_i}
\]
A quantity $\sum_i p_i(\xx) \dx{x_i}$ is an exact differential if and only if 
\[
\frac{\partial p_i}{\partial x_j} = \frac{\partial p_j}{\partial x_i}.
\]
This means that 
\[
\frac{\partial F_kc_{ki}}{\partial F_j} = \frac{\partial F_kc_{kj}}{\partial F_i} 
\]
\[c_{ji} = c_{ij}\]
$\cc$ is symmetric.

This means that if we apply a point force a point $i$ and measure displacement at $j$, we get the same displacement as if we apply the force at $j$ and measure displacement at $i$.
$\cc = \cc^T$ is known as Maxwell-reciprocity. This is analogous to self-adjointness of ODEs/PDEs. e.g. if I have an ODE
\[
\LL[y] = f(x)
\]
I can construct a Green's function solution
\[
\LL_x[\GG(x,\tilde x )] = \delta(x-\tilde x )
\]
If $\LL$ is a self-adjoint operator then 
\[
\GG(x, \tilde x) = \GG(\tilde x, x).
\]
The PDEs for linear elasticity theory are self-adjoint.

\begin{enumerate}
\item
Suppose we load (i.e. apply a set of forces) to an elastic body doing an amount of work $U = \frac 1 2 \sum_{i,j}c_{ij}F_iF_j$.
Now we take the forces away; in remaining the forces, we must do work $-U$ i.e. the forces within the body do work $U$. we can think of $U$ as the {\bf internal elastic energy} within a deformed body.
 
 \item {Lorentzian or Betti-Rayleigh reciprocity.}
 Consider the same body deformed under two different set of forces $\{F_i\}, \{F_i'\}$with reps displacements $\{u_i\}, \{u_i'\}$. Index notation below
 \[
F_i'u_i = F_i' c_{ij}F_j =  F_i' c_{ji}F_j = F_i c_{ij}F_j' = F_i u_i' 
 \]
 This is known to a bunch of Harvard people (including Roper) as {\bf something for nothing formula}.
 
 \item {Uniqueness.} Using $\cc$ I can calculate $\{u_i\}$ from $\{F_i\}$ using $\uu  = \cc\cdot\FF$
 I can similarly calculate $\FF$ (uniquely) from $\uu$ as 
 \[
\FF = \kk\cdot \uu
 \]
 where $\kk= \cc^{-1}$ and $\cc$ is invertible because it is (by Hooke's first law).
 $\kk$ is called the matrix of spring constants.
 \[
 U= \frac 1 2 \sum_i F_i u_i = \frac 1 2 \sum_{ij} k_{ij}u_iu_j
 \]
 
 
 \item Principle of virtual work.
 \[
\frac{\partial U}{\partial u_i} = \sum_j k_{ij}u_j = F_i
 \]
 Take a stressed and already deformed body; and imagine applying additional virtual displacements $\{\delta u_i\}$ to the body. The external forces $F_i$ do ``virtual work": external virtual work (EVW) $ \sum_i F_i\delta u_i$.
 
 I also change the elastic (internal) energy of the body by 
 \[
\delta U = \sum_i \frac{\partial U}{\partial u_i} \delta u_i
 \]
 also known as "internal virtual work" (IVW). The idea IVW = EVW is key to Finite Element Models for linear elasticity. The principle of virtual work is often assumed and can be used to solve problems with {\bf large deformations}, membranes, etc.
\end{enumerate}

\bigskip
\bigskip
Often our b.c.s may consist of a combination of known forces and known displacements. A loaded bridge has $\uu=0$ at points of support. Is he displacement field unique when we use mixed b.c.s.?

Assume we have two set of b.c.s. $\cF$ where $F_i$ is know, $\UU$ where $u_i$ is known, $\cF$ and $\UU$ disjoint.

Can we specify a set of forces $F_i^*$ such that 
\[
F_i^*  = F_i \quad  \forall i\in \cF
\]
\[
\sum_jc_{ij}F_j^*  = u_i \quad \forall i\in \UU
\]

We need to solve $|\UU|$ equations to calculate $|\UU|$ unknown $F_i^*$
\[
\sum_{j\in \UU} c_{ij}F_j^* = u_i - \sum_{j\in\cF} c_{ij}F_j^*
\]

Because $\cc$ is positive definite, all of its principal minors are positive. We can order the indices $i$ so that the first $|\UU|$ indices are associated with $\UU$ and the remaining indices with $\cF$.
Because $[\cc]\in\UU\times\UU$ has positive determinant, the above equation can be uniquely solved.

Method for finding approximate (even exact) solutions to equations.
Call a set of forces $\{F_i^*\}$ admissible if $\{F_i^*\} = F_i$ $\forall i\in\cF$.
Claim: the unique admissible solution of minimized the complementary energy
\[
V(F_i^*) = U(F_i^*) - \sum_{i\in\UU} F_i^*u_i
\]
(this is a Legendre transform.)

To minimized $V$ we need $\frac{\partial V}{\partial F_i^*}=0$ $\forall i\in\UU$:
\[
\sum_jc_{ij}F_j^* - u_i=0 \quad \forall i\in\UU.
\]
Still need to show that $\{F_i^*\}$ is a global minimizer. To do this we show that $V\to \infty$ in all directions.
By positive definiteness of $\cc$, 
\[
\sum_{ij} c_{ij}  F_i^* F_j^* \geq \lambda_{\min}\|\FF\|^2,
\]
whereas
\[
|\sum_{i\in\UU}u_iF_i^*| \leq u_{\max} \|\FF\|.
\]
So $V\to\infty$ as $\|\FF\|\to\infty$.

This is known as Principle of minimum complementary energy.

We can instead write down our problem in elasticity theory. 
We specify $u_i$, $i\in\UU$, and we want to solve for  $u_i$, $i\in\cF$. There will be unique $u_i$, $i\in\cF$ satisfying all equations; and if we define admissible displacements $u_i^*$ such that  $u_i^* = u_i$ $\forall i\in\UU$ and $u_i^*$ unconstrained $\forall i\in\cF$ and the unique solution to the problem minimizes 
\[
V(u_i^*) = U(F_i^*) - \sum_{i\in\cF} F_i^*u_i
\]

\section{Tensors}
\subsection{Cartesian tensors and vector calculus \& summation convention}
 Components of vectors are related via transformation/rotation matrices.
\[
l_{ij} = \ee_i'\cdot\ee_j.
\]
If we have a vector $\FF$ whose components are $F_i$ in the ``old" coordinate system,
\[
\FF = F_i\ee_i
\]
using summation convention. The components of $\FF$ in the new coordinate system are
\[
\FF = F_i'\ee_i',
\]
where 
\[
F_i' = \FF\cdot\ee_i'= F_j\ee_j\cdot\ee_i'
\]
\begin{equation}\label{eq:3.1}
F_i' = l_{ij} F_j.
\end{equation}
A vector in $\RR^3$ is a set of 3 quantities $\FF = \mx{F_1\\F_2\\F_3}$ that when the coordinate system is changed, transforms according to \ref{eq:3.1}. Examples include forces, velocitiy, displacements, position, torque.

we will define a rank $n$ tensor $\underline{\underset{\vdots}{\underline\bA}}$ ( underlined $n$ times) that is made up of $3^n$ components in $\RR$, $A_{i_1, i_2, \dots,  i_n}$. When we transform coordinates, these components will transform in a way that generalize \ref{eq:3.1}:
\[
A_{i_1, i_2\dotsi_n} = l_{i_1j_1} l_{i_2j_2} \cdots l_{i_nj_n}A_{j_1, j_2\dots j_n}.
\]
e.g. 

$n=0$, $A$ is s scalar. e.g., temperature, pressure, energy.

$n=1$, $\underline A$ is a vector that transforms by matrix multiply.

$n=2$, $\underline{\underline A}$ is an array of 9 numbers that transforms by 
\[
A' = LAL^T, \quad L = (l_{ij}),
\]
\[
A' = l_{ik} l_{jm} A_{km} =  l_{ik} A_{km} (l^T)_{mj} 
\]
where $A$ is an old set of 9 numbers, $A'$ is the new set of 9 numbers. e.g. include: stress, strain (or rate of strain).

\begin{theorem} (see Landau \& Lifschitz)
Define an {\bf isotropic} tensor to be a tensor $\underline{\underline A}$ whose coordinate components are independent of the {\bf allowed} choice of coordinate system.
\end{theorem}
In $\RR^3$,

$n=0$, all scalars are isotropic.

$n=1$, $\bf 0$ is the only isotropic vector.

$n=2$, the only isotropic tensors have components, 
\[
A_{ij} = \lambda \delta_{ij}.
\]

$n=3$, only isotropic tensor is the alternating tensor
\[
\lambda\epsilon_{ijk}.
\]

$n=4$, $\lambda\delta_{ij}\delta_{kl}$,  $\mu\delta_{ik}\delta_{jl}$,  $\nu\delta_{il}\delta_{jk}$ and linear combinations thereof.

\bigskip
\bigskip

{\bf Important identity}
\[
\epsilon_{ijk}\epsilon_{ilm} = \delta_{jl}\delta_{km} - \delta_{jm}\delta_{kl}.
\]

Using summation convenction to evaluation some vector calculus quantities.
\[
\bnabla(\frac{\xx}{r}), \text{ where } r = \sqrt{x^2 + y^2 + z^2}.
\]
\[
[\bnabla(\frac{\xx}{r})]_{ij}= \frac{\partial}{\partial x_i}\frac{x_j}{r} = \frac{\delta_{ij}}{r} - \frac{x_ix_j}{r^3}
\]
\[
[\bnabla(\frac{\xx}{r})] = \frac{\bnabla \xx}{r} + \frac{\xx\xx}{r^3}
\]






\subsection{Multiplication of vectors}
Given 2 tensors $\bA$ (rank n) and $\bB$ (rank m), we can define different products of $\bA$ and $\bB$.

Outer product:
\[
\bC  = \bA \bB
\]
\[
C_{i_1 i_2 \dots i_{m+n}} = A_{i_1i_2\dots i_n} B_{i_{n+1}\dots i_{n+m}}.
\]

Inner product: 
\[
\bC = \bA \cdot\bB 
\]
\[
C_{i_1 i_2 \dots \i_{m+n-2}} = A_{i_1i_2\dots i_{n-1}j} B_{j i_n i_{n+1}\dots i_{n+m-2}}.
\]

And we can keep summing, contraction: 
\[
\bC = \bA : \bB
\]
\[
C_{i_1 i_2 \dots i_{m+n-4}} = A_{i_1i_2\dots i_{n-2}jk} B_{k j i_{n-1} i_{n}\dots i_{n+m-4}}.
\]

Given two vectors $\bA$, $\bB$ we can multiply in 3 ways.
\[
\bA\cdot\bB, \quad \bA\bB, \quad \bA\times\bB
\]
\[
[\bA\times\bB]_i = \epsilon_{ijk}A_jB_k
\]
\[
\bA\times\bB = \epsilon:(\bB\bA)
\]
We can define the Kronecker (identity/idempotent) tensor
\[
[\mathds 1 ]_{ij} = \delta_{ij}
\]
\[
\bA\bB = \delta_{ij}A_iB_j = {\mathds 1}:(\bA\bB)
\]

Other methods of multiplying do not produce tensors (for example taking component-wise product of two tensors).
If I have another tensor $\cc$, I could form a product $\bA\cdot\cc\cdot\bB$, but this is a product between $\bA$, $\cc$, and $\bB$.
\begin{enumerate}
\item{Example}

Consider, $\uu$ a vector field, $\uu: \RR^3 \to\RR^3$,
\begin{align*}
[\uu\times(\nabla\times\uu)]_i &= \epsilon_{ijk}u_j(\nabla\times\uu)_k = \epsilon_{ijk}u_j\epsilon_{klm}\frac{\partial u_m}{\partial x_l}\\
&=  \epsilon_{kij}\epsilon_{klm} u_j\frac{\partial u_m}{\partial x_l}\\
&=(\delta_{il}\delta_{jm} - \delta_{im}\delta_{jl})u_j\frac{\partial u_m}{\partial x_l}\\
&=u_m\frac{\partial u_m}{\partial x_i} - u_l\frac{\partial u_i}{\partial x_l}\\
&=\bnabla (\frac 1 2 \uu^2) - (\uu\cdot\bnabla)\uu
\end{align*}
\item{Example}

Say that a tensor $\bA$ is symmetric if $A_{ij} = A_{ji}$, and anti-symmetric if $A_{ij} =- A_{ji}$. Suppose $\bS$ is symmetric, $\epsilon:\bS = 0$ by simple calculation.
\end{enumerate}
We can write vectors in the form
\[
\FF = F_i\ee_i,
\]
where $\{\ee_i\}$ are our basis vectors. We can write a tensor in the same way
\[
\bA = A_{ij}\ee_i\ee_j
\]
The outer products $\ee_i\ee_j$ is a bsis for rank 2 tensors.

We can write tensors w.r.t. $\hat\xx, \hat{\bf y}, \hat{\bf z}$, or w.r.t. $\ee_r, \ee_{\theta}, \ee_{\phi}$ (spherical polar basis), or  $\ee_r, \ee_{\phi}, \ee_{z}$ (cylindrical polar basis).




\subsection{Stress Tensor}
Consider a body that is deformed by applying some set of forces. Before the forces are applied, I can identify each point in the body by its position $\bX$. I will assume that there is a differentiable field $\xx (\bX,t)$. $\xx (\bX,t)$ tells me the new location of the point at time $t$. So we can define a displacement field $\uu$, $\uu = \xx-\bX$, and a vector field $\ff(\xx,t)$ that defines the force per unit volume of material. 

Suppose we have an infinitesimal volume $\dV$, located at $\xx$, the total body force acting on $\dV$ is $\ff(\xx,t)\dV$. Body forces include gravity, electric forces. We also need to define a {\bf traction} or {\bf surface force field} as follows. 

Across any element of surface has unit normal $\nn$, and area $\dS$,  {\bf traction forces} must act the material on the positive side of the element applies a force $\bT_{\nn}\dS$ on the material on the negative side of the element.

That is, I assume I can define a vector field $\bT(\xx, t)$ that tells me what elastic forces are acting within the body. In fact, we can show that all of the traction fields associated with the deformation of a body can be derived from a single second rank tensor field $\ssigma(\xx,t)$, that is
\[
\bT_{\nn}(\xx,t) = \ssigma(\xx,t)\cdot \nn .
\]
By Newton's third law,
\[
\bT_{\nn}(\xx,t) = -\bT_{-\nn}(\xx,t).
\]

Consider Cauchy's stress tetrahedron, $\nn = (n_x, n_y, n_z)$. Length scale $L\to 0$ eventually.
mass $\times$ acceleration $O(L^3)$= sum of traction forces $O(L^2)$ $+$ body forces $O(L^3)$. As $L\to 0$, sum of traction forces must be 0.  Force on sloping face:
\[
\bT_{\nn}\rvert_ {\text {centroid of face}}\dS = (\bT_{\nn}\rvert_{\{0,0,0\}}+ O(L))\dS 
\]
Similarly for other faces. By force balance:
\[
\bT_{\nn}\dS + \bT_{-\ee_z}n_z\dS +\bT_{-\ee_y}n_y\dS +\bT_{-\ee_x}n_x\dS = 0,
\]
so 
\[
\bT_{\nn}\dS = \bT_{\ee_z}n_z\dS +\bT_{\ee_y}n_y\dS +\bT_{\ee_x}n_x\dS .
\]

Once $\bT_{\ee_x}$, $\bT_{\ee_y}$,$\bT_{\ee_z}$, are all known, we can calculate $\bT_{\nn}$, for arbitrary $\nn$. We can construct a tensor from these vectors.
\[
\sigma_{11} = (\bT_{\ee_x})_1, \quad \sigma_{12} = (\bT_{\ee_y})_1,\quad \sigma_{13} = (\bT_{\ee_z})_1
\]
that is, 
\[
\bT_{\nn}(\xx,t) = \ssigma(\xx,t)\cdot \nn .
\]
for an array of $3\times 3$ numbers. is it  rank 2 tensor.
Note that $\bT_{\nn}$ and $\nn$ are both vectors. Let's suppose we transform coordinates, $\nn$ has new components $n_i'$ ,
\[
{T_{n}}'_i = l_{ij} {T_{n}}_j = l_{ij} \sigma_{jk}n_k
\]
\begin{align*}
\sigma'_{ik}n'_k &=  l_{ij} \sigma_{jk} l_{mk}n_m' \quad (n_k = l_{mk}n_m')\\
&=   l_{ij} \sigma_{jm} l_{km}n_k' 
\end{align*}
\[
n_k'(\sigma_{ik}' - l_{ij}l_{km}\sigma_{jm})=0.
\]
for arbitrary $n_k'$ vectors. This shows that the $\ssigma$ we defined is indeed a tensor.

Take a control volume at equilibrium,
\[
\int_V \ff \dV + \int_S \bT_{\nn}\dS = 0.
\]
By  divergence theorem, which says that
\[
\int_S \sigma_{ijkl}n_i = \int_V \frac{\partial \sigma_{ijkl}}{\partial x_i},
\]
we have
\[
\int_V \ff \dV + \int_V\frac{\partial (\sigma_{ij}\ee_i)}{\partial x_j}\dV = 0.
\]
\[
\ff + \nabla \cdot\ssigma^T = 0.
\]
(except possibly on a set of measure 0.)
This is called {\bf Cauchy's equilibrium law}.

Now consider the torque,
\[
\int_V \rr\times \ff \dV + \int_S (\rr \times \ssigma\cdot \nn)\dS = 0.
\]
(next steps do not apply to magnetic objects).
\begin{align*}
\int_S \ee_i \epsilon_{ijk}x_j\sigma_{kl}n_l\dS &=  \int_V \ee_i \epsilon_{ijk}\frac{\partial}{\partial x_l}(x_j\sigma_{kl}) \dV\\
&= \int_V \ee_i \epsilon_{ijk} ( \delta_{jl}\sigma_{kl} + x_j\frac{\partial}{\partial x_l}\sigma_{kl}) \dV\\
&=  \int_V \ee_i \epsilon_{ijk} ( \sigma_{kj} + x_j\frac{\partial}{\partial x_l}\sigma_{kl})\dV
\end{align*}

\[
\int_V (\rr\times(\ff+\nabla\cdot\ssigma^T) + \ee_i\epsilon_{ilk}\sigma_{kl}=0.
\]
The first term vanishes by force balance. The second term tells us that 
\[
\ee_i \epsilon_{ijk}\sigma_{kj} = 0,
\]
so 
\[
\epsilon_{ijk}\sigma_{kj} = 0 \quad \forall i.
\]
i.e.
\[
\ssigma = \ssigma^T.
\]

Need to relate $\ssigma$ to how much the body deforms.








\subsection{Measuring deformation/strain}
We are interested in knowing how far initially nearby parts of the object move apart when it is deformed.  A point at $\bX + \dx{\bX}$ move to $\xx + \dx{\xx}$, the original separation is $\dx{S}^2 = \dx{X_i}\dx{X_i}$, are now separated by $\dx{s}^2 = \dx{x_i}\dx{x_i}$.
If $\dx{\bX_i}$ and $\dx{\xx_i}$ are infinitesimal,
\[
\dx{x_i}  = \frac{\partial x_i}{\partial X_j} \dx{X_j}.
\]
\begin{align*}
ds^2 - dS^2 &= ( \frac{\partial x_i}{\partial X_j}  \frac{\partial x_i}{\partial X_k}  - \delta_{jk} )\dx{X_j} \dx{X_k} \\
&:=  2 e_{jk} \dx{X_j} \dx{X_k}\\
&= ( \delta_{jk} -  \frac{\partial X_i}{\partial x_j}  \frac{\partial X_i}{\partial x_k} )\dx{x_j}\dx{x_k}\\
&:=  2 E_{jk} \dx{x_j} \dx{x_k}.
\end{align*}
$\underline{\underline e}$ is called Cauchy-Almansi strain, and $\underline{\underline E}$  Green's strain.

It is useful to rewrite in terms of displacement:
\[
\uu = \xx - \bX
\]
\[
e_{jk} = \frac 1 2 ( \frac{\partial u_i}{\partial X_j} \frac{\partial u_i}{\partial X_k} +  \frac{\partial u_j}{\partial X_k}  +\frac{\partial u_k}{\partial X_j}  )
\]

\[
E_{jk} = \frac 1 2 ( \frac{\partial u_k}{\partial x_j}  +  \frac{\partial u_j}{\partial x_k}  - \frac{\partial u_i}{\partial x_j} \frac{\partial u_i}{\partial x_k}      )
\]


\section{Linear Elasticity}
Suppose that the deformation of the body is small, i.e. $\|\bnabla \uu\| \ll 1$, and assume displacement is small, $\| \uu\|\ll L$, where L is characteristic object size. Then to leading order the strain tensors have the same components
\[
E_{ij} \approx e_{ij} \approx \frac 1 2 (\frac{\partial u_i}{\partial x_j} + \frac{\partial u_j}{\partial x_i})
\]

This is called Cauchy's infinitesimal strain tensor.

Interpretation. Consider a line element aligned with the x-axis. We will see that line element change in length is 
\[
ds^2 - dS^2 = 2 e_{xx}dX^2.
\]
Write
\[
\uu = (u, v, w),
\]
\begin{align*}
\dx{s^2} &= \dx{x^2} + \dx{y^2} + \dx{z^2} \\
 &= (X + dX + u(X+dX, Y, Z) - X- u(X,Y,Z))^2  \\
 &+   (v(X+dX,Y , Z)  - v(X,Y,Z))^2 \\
 &+  ( w(X+dX, Y, Z) - w(X,Y,Z))^2\\
 &= dX^2 (1+2\frac{\partial u}{\partial X} ) + h.o.t \quad (\|\nabla \uu \|\ll 1)
\end{align*}
Change in length is
\[
ds^2 - dS^2 = 2 \frac{\partial u}{\partial X} dX^2 = 2 e_{xx} dX^2
\]
Relative change in length
\[
\frac{ds-dS}{dS} = \sqrt{1+2e_{xx}}  -1
\]
$e_{xx}$ measures {\bf fractional stretch} in the x-direction.

Now consider the off-diagonal terms, for example $e_{xy}$.
\[
A = (u(0), v(0),w(0))
\]
\[
B = (u(0), v(0),w(0)) + dX(1+ \frac{\partial u}{\partial X},  \frac{\partial v}{\partial X},  \frac{\partial w}{\partial X})
\]
\[
C = (u(0), v(0),w(0))+ dX( \frac{\partial u}{\partial X},  1+\frac{\partial v}{\partial X},  \frac{\partial w}{\partial X})
\]

\[
\tan \alpha_x = \frac{ \frac{\partial v}{\partial X}}{1 +  \frac{\partial u}{\partial X}}
\]
\[
\alpha_x\approx  \frac{\partial v}{\partial X}.
\]
Similarly,
\[
\alpha_y\approx  \frac{\partial u}{\partial X}.
\]
Change in the angle made by the two sides of the $L$ is 
\[
-\alpha_x -\alpha_y = -2e_{xy}.
\]
Why does $\underline{\underline e}$ characterize strain but not $\bnabla\uu$ or $(\bnabla\uu)^T$?\[
\underline{\underline e} = \frac 1 2 (\bnabla\uu + (\bnabla\uu)^T)
\]
Let's look at the displacements within a body near a point {\bf 0}. If a point start at $\dx{\bX}$, and gets moved to $\uu(\dx{\bX} )$, I expect the body to resist elastically.
\[
\uu(\dx{\bX}) - \uu(0) = (d\bX \bnabla) \uu(0) = d\bX\cdot(\bnabla \uu).
\] 
Decompose $\uu$ into symmetric and skew-symmetric parts. The symmetric part is $\te$, call the skew-symmetric part $\ta$. Let
\[
\omega_i = \frac 1 2 \epsilon_{ijk}a_{jk}
\]
Then we claim that 
\[
a_{ij} = \epsilon_{ijk}\omega_k.
\]
Proof of claim:
\begin{align*}
\epsilon_{ijk}\omega_k &= \frac 1 2  \epsilon_{ijk}  \epsilon_{klm} a_{lm} \\
&=  \frac 1 2  \epsilon_{kij}  \epsilon_{klm} a_{lm}\\
&=\frac 1 2 (\delta_{il}\delta_{jm} - \delta_{im}\delta_{jl})a_{lm}\\
&= \frac 1 2 (a_{ij} - a_{ji}) \\
&= a_{ij}.
\end{align*}
\[
dX\cdot \ta = \omega\times dX
\]
This is rigid body rotation.


$\te$ is a symmetric tensor, i.e. is diagonizable wrt three perpendicular eigenvectors $\vv_1,$ $\vv_2,$ $\vv_3$ , where $\te\cdot \vv_1 = \lambda_1\vv_1$. Call the $\vv_i$'s the {\bf  directions of principal stretch}.




\subsection{Constitutive equation for linear elasticity}
A constitutive equation gives us a way of computing stresses from strains
\[
\ssigma  = f(\te)
\]
$f$ is material specific.

Want $\ssigma$ to be linearly dependent on $\te$. 
i.e. we assume
\[
\sigma_{ij} = C_{ijkl}e_{kl}
\]
Where $C$ is a rank 4 tensor, (the {\bf elastic modulus tensor})
\[
C_{ijkl} = C_{jikl}
\]
because $\ssigma$ is symmetric.
$\te$ is also symmetric, so only symmetric part of $C_{ijkl}$ contributes to $\sigma_{ij}$, so assume $C_{ijkl} = C_{ijlk}$ wlog. So $C_{ijkl}$ has 36 free components.

We can reduce the number of free components further by making an additional assumption. P.V.W.

Calculate total elastic energy stored in a control volume V, when body is deformed. $\ssigma$ and $\FF$ are ramped up slowly from 0 to $\ssigma^*$, $\FF^*$. Work done in increasing forces 
\begin{align*}
\dx{U} &= \int_V \FF\cdot\dx{\uu} \dx{V} + \int_{\partial V} \dx{u}\cdot \ssigma\cdot\nn\dx{S}\\
&= \int_V \FF\cdot\dx{\uu} + \nabla\cdot (\dx{\uu}\cdot \ssigma) \dx{V} \\
&= \int_V \FF\cdot\dx{\uu} + \dx{\uu}\cdot(\nabla\cdot\ssigma^T) + \nabla\dx{\uu}:\ssigma\dx{V}\\
&= \dx{e_{ij}}\sigma_{ij} 
\end{align*}
where $\dx{\te} = \frac 1 2 (\nabla\dx{\uu} + (\nabla\dx{\uu} )^T)$. The first two terms vanish by force balance, the third terms reduces by symmetry of $\ssigma$.

\[
dU = \int_V\sigma_{ij}\dx{e_{ij}}dV
\]
total amount of work that we do is 
\[
U = \int_V\int_0^{\te^*}\sigma_{ij}\dx{e_{ij}}dV
\]
We can define a scalar field 
\[
U = \int_{strain} \sigma_{ij}\dx{e_{ij}}
\]
which gives elastic energy stored volume of body.
\[
U = \int_0^1C_{ijkl}e_{kl}^*te_{ij}^*dt = \frac 1 2 C_{ijkl}e_{ij}^*e_{kl}^*
\]

Total work done per volume in building up stress $\sigma_{ij}$ is $\frac 1 2 C_{ijkl}e_{ij}e_{kl} = \frac 1 2 \sigma_{ij}e_{ij}$ if stress increases linearly in time.

If this quantity is independent of the way stress ( + strains) are built up, then we can define PVW
\[
\frac{\partial U}{\partial e_{ij}} = \sigma_{ij}
\]
so
\[
\sigma_{ij} = \frac 1 2 C_{ijkl}e_{kl} + \frac 1 2 C_{klij}e_{kl}
\]
\[
C_{ijkl}e_{kl} = C_{klij}e_{kl}
\]
So
\[
C_{ijkl} = C_{klij}
\]
i.e. $C_{ijkl}$ as at most 21 independent components.

Most materials we study are isotropic. i.e.
\[
C_{ijkl} = \lambda \delta_{ij}\delta_{kl} + \mu \delta_{ik}\delta_{jl}  + \nu \delta_{il}\delta_{jk}
\]
for 3 coefficients $\lambda$, $\mu$, $\nu$.
\[
\Rightarrow \sigma_{ij}  = C_{ijkl}e_{kl} = \lambda \delta_{ij}e_{kk} + \mu e_{ij} + \nu e_{ji} := \lambda\delta_{ij}e_{kk} + 2Ge_{ij}
\]
$\lambda$ and $G$ are called {\bf Lame coefficients}. Now we want to express $\te$ in terms of $\ssigma$. 
\[
\sigma_{mm} = \lambda e_{kk} + 2G e_{mm}
\]
so
\[
e_{mm} = \frac {1}{\lambda + 2G} \sigma_{kk}
\]
\begin{align*}
e_{ij} &= \frac{1}{2G} (\sigma_{ij} - \frac{\lambda\delta_{ij}\sigma_{kk}}{3\lambda+ 2 G})\\
	&= \frac1 E ((\nu+1)\sigma_{ij} -\nu \sigma_{kk}\delta_{ij})\\
\end{align*}
E is called {\bf Young's modulus}.
$\nu$ {\bf Poisson ratio}.
$\lambda$, $G$, $E$ are all elastic moduli.
\[
[\sigma] = Nm^{-2} = Pa
\]
\[
[\nu] = 1
\]
\[
[G], [\lambda], [E]= Pa
\]

Pull a block of elastic object in the $x$-direction, expected stress in block is
\[
\sigma_{11} = \sigma, \sigma_{ij} = 0 \text{ otherwise}
\]
\[
e_{11} = \frac 1 E ((\nu+1)\sigma - \nu\sigma) = \sigma/E
\]
$E$ tells us how much stretch there is in the direction in which the force is applied.
\[
e_{22} = e_{33} = -\frac{\nu\sigma}{E}
\]
\[
\frac{e_{22}}{e_{11}} = -\nu
\]
$\nu$ gives ratio of shrinkage in the perpendicular direction to stretch in direction parallel of forces.

\[
-1\leq\nu\leq \frac 1 2
\]
$\nu = \frac 1 2 $ perfectly preserves volume. 

Now in a different scenario, apply forces in $x$ and $y$-direction.

\[
\sigma_{11}  = \sigma_1, \sigma_{22} = \sigma_2
\]
\[
e_{11} = \frac 1 E ((\nu+1)\sigma_1 - \nu(\sigma_1 + \sigma_2))  = \frac 1 E (\sigma_1 - \nu\sigma_2)
\]
If $\nu>0$, then biaxial force $\sigma_2$ reduces stretch in the x-direction.


Now shearing a block of object between two plates.
\[
\sigma_{12} = \tau
\]
\[
\sigma_{12} = 2Ge_{12}
\]
\[
e_{12} = \sigma_{12}/2G = \tau/2G
\]
If plates are separated by $w$, and total shift is $\Delta x$, then 
\[
e_{12} = \frac 1 2 (\frac{\partial u }{\partial y } + \frac{\partial v}{\partial x} ) = \frac 1 2 \frac{ \Delta x }{ w} 
\]
\[
 \frac{ \Delta x }{ w} = \frac \tau G
\]
\[
G = \tau / (\frac{\Delta x }{w})
\]
G is the shear modulus.

\subsection{PDEs of linear elastic bodies in equilibrium}
Cauchy's law
\[
\bnabla\cdot\ssigma  + \FF = 0.
\]
Now 
\[
\ssigma = \lambda \bnabla\cdot\uu \mathds{1} + G(\bnabla\uu + (\bnabla\uu)^T)
\]
\[
(\lambda + G)\bnabla(\bnabla\cdot \uu) + G\nabla^2\uu + \FF = 0
\]
This is called {\bf Navier's equation}.

Modification for non-equilibrium.
\[
mass\times acceleration  =\int_V\FF\dx{V} + \int_{\partial V}\ssigma\cdot\nn\dx{S}
\]

\[
\int_V\rho\frac{\partial^2 u}{\partial t^2} \dx{V} =\int_V\FF\dx{V} + \int_{\partial V}\ssigma\cdot\nn\dx{S}
\]
$\rho$ is mass per unit volume in the undeformed state.
\[
\int_V\rho\frac{\partial^2 u}{\partial t^2} \dx{V} = (\lambda + G)\bnabla(\bnabla\cdot \uu) + G\nabla^2\uu + \FF .
\]
This is a wave equation.

Assume $\FF=0$ and the body is in equilibrium. take divergence
\[
(\lambda + G)\bnabla^2(\bnabla\cdot \uu) + G\nabla^2(\nabla\cdot\uu)=0
\]
\[
\Rightarrow \nabla^2(\nabla\cdot\uu)=0
\]
So $e_{kk}$ is a harmonic function.
\[
\Rightarrow \sigma_{kk} = (3\lambda + 2G)e_{kk}
\]
So $\sigma_{kk}$ is also harmonic.

Also take $\nabla^2$ on Navier's equation, $\nabla^2(\nabla^2\uu)=0$, so $\nabla^4\uu=0$. i.e. $\uu$ is {\bf biharmonic}. 

\subsubsection{}
The Navier equation has a unique solution in a domain $\Omega$, if $\uu$ is specified on the boundary $\partial\Omega$, or if $\partial \Omega$ can be disjointly partitioned into $\partial\Omega_u$ ($\uu$ is speficied) \& $\partial \Omega_F$ ($\ssigma\cdot\nn$ are specified).

Recall Helmholtz's theorem, exists $\phi$ and $\bA$ such that
\[
\uu = \bnabla\phi + \bnabla\times\bA
\]
\[
e_{kk} = \nabla\cdot\uu = \nabla^2\phi
\]
\[
\nabla\times\uu =  \bnabla\times(\bnabla\times\bA) = -\nabla^2\bA
\]
If $\nabla\times\uu=0$, then we say $\uu$ is irrotational. i.e. $\uu = \bnabla\phi$ (ansatz).

Consider a spherical capsule, with a difference in pressure between interior and exterior. Calculate the deformation. Assume spherical symmetric deformation. Define spherical polar coordinates $(r, \theta, \phi)$
\[
\uu = u(r)\ee_r.
\]
Assume there is a potential $\phi(r)$ with $\uu = \nabla\phi$.
At $r = a$, 
\[
-\ssigma\cdot\ee_r = \Delta p \ee_r
\]
\[
\sigma_{rr} = -\Delta p , \sigma_{r\theta} =\sigma_{r\phi}= 0
\]
\[
\ssigma\cdot\ee_r =  -  p_{outer} \ee_r
\]

Navier's equations and $\uu = \nabla\phi$,
\[
\bnabla(\nabla^2\phi)=0.
\]
\[
\nabla^2\phi= A
\]
for some constant $A$.
\[
\frac{1}{r^2\sin\theta}\frac{\partial}{\partial r}(r^2\sin\theta\frac{\partial\phi}{\partial r})= A.
\]
\[
\frac{1}{r^2}\frac{\partial}{\partial r}(r^2\frac{\partial\phi}{\partial r})= A.
\]
\[
r^2\frac{\partial\phi}{\partial r} = Ar^3 + B
\]
\[
\phi(r) = Ar^2 + B/r + C
\]



$\sigma_{\theta\theta}$ and $\sigma_{\phi\phi}$ are known as hoop stresses. (see beam/elastic theory in this course.)

\subsubsection{}
In equilibrium
\[
\bnabla\cdot\ssigma + \FF = {\bf 0}.
\]
$\ssigma$ has 6 components, but  only have 3 equations.

To get Cauchy's equation, we need to add a constitutive law
\[
\ssigma = \lambda e_{kk} \mathds 1 + 2G\te
\]
but
\[
\te = \frac 1 2 (\bnabla\uu + (\bnabla\uu)^T)
\]
In general (see homework \#2), $\te$ can be derived from a real displacement field iff the St. Venant compatibility conditions hold.
\[
e_{ij,kl} + e_{kl,ij} - e_{ik,jl} - e_{jl,ik}=0
\]
Recall
\[
e_{ij} = \frac{(1+nu)\sigma_{ij}}{E} - \frac{\nu\sigma_{kk}\delta_{ij}}{E}
\]
and substitute into the previous equation
\[
(1+\nu)(\sigma_{ij,kl} + \sigma_{kl,ij} - \sigma_{ik,jl} - \sigma_{jl,ik}) = \nu(\sigma_{mm,kl}\delta_{ij} + \sigma_{mm,ij}\delta_{kl} - \sigma_{mm,kl}\delta_{ik} - \sigma_{mm,kl}\delta_{jl} )
\]
Let $k=l$,
\[
(1+\nu)(\nabla^2\sigma_{ij} + \sigma_{mm,ij} - (\nabla\ssigma)_{i,j}  - (\nabla\ssigma)_{j,i} ) = \nu (\nabla^2\sigma_{mm,ij} + \sigma_{mm,ij})
\]
I have reduced S-V to 6 equations.
\[
\nabla^2\sigma_{ij} + \frac{1}{1+\nu}\sigma_{mm,ij} = \frac{-\nu}{1-\nu}(\bnabla\cdot\FF)\delta_{ij} - (F_{i,j}+ F_{j,i})
\]
Beltrami-Mitchell compatibility equations.
If we take Cauchy's equation and add (the right) three of the Beltrami-Mitchell equations, we can calculate $\sigma_{ij}$ uniquely.


Contract  $i=j$,
\[
(1+\nu)\nabla^2\sigma_{mm} + \nabla^2\sigma_{mm} - 3\nu\nabla^2\sigma_{mm} = -(1+\nu)2\bnabla\cdot\FF
\]
\[
2(1-\nu)\nabla^2\sigma_{mm} = -2(1+\nu)\nabla\cdot\FF
\]
Example: consider a beam, ($\nn\cdot\ssigma$ on other surface.)

boundary conditions are that $\sigma_{xx}   = \alpha z $ at the two ends of the beam.
Solve in terms of the stress field.
I assert that $\ssigma = \mx{\alpha z & 0 & 0\\ 0&0&0\\0&0&0}$ solves Cauchy's equilibrium and Beltrami-Mitchell and solves the b.c.s. 
\[
\bnabla\cdot \ssigma = \mx{\frac{\partial \sigma_{xx}}{\partial x}+ \frac{\partial \sigma_{yx}}{\partial y} + \frac{\partial \sigma_{zx}}{\partial z} \\...\\...} = 0.
\]
In absence of body forces, B-M is satisfied.

\[
\te = \frac{1+\nu}{E}\ssigma - \frac{\nu\sigma_{kk}}{E}\mathds 1 
\]
\[
e_{xx} = \frac{\sigma_{xx}}{E} = \frac{\alpha z}{E}
\]
while $e_{zx}=0$, right angles remain right angles; so beam bends into the arc of a circle. center line (surface $z=0$) is unstreched by bending. at distance $h$ from the center line, the surface is stretched by factor of $\frac{\alpha h}{E}$. If initial length of beam is $L$, then
\[
R\theta = L
\]
\[
(R+h)\theta = (1+\alpha h/E)L
\]
\[
\Rightarrow R = \frac E \alpha
\]
Let's assume no net force is being applied.
\[
\int_{cross-section} \sigma_{xx}\dx{A}=0
\]
\[
\Rightarrow \int z \dx{A} = 0
\]
$z=0$ must pass through the beam centroid. Call $z=0$ the {\bf neutral surface}.
Define a moment of force,
\[
M = \int_{cross-section} z\sigma_{xx}\dx{A} = \alpha \int_{cross-section}z^2\dx{A}:= \alpha I
\]
\[
R = \frac{E}{M/I}
\]
\[
M = EI/R
\]





\subsection{Plane strain \& plane stress}
There are two types of 2D elasticity problem. Thick samples under forces that are homogeneous in $z$ and are oriented in $x$, $y$ plane. Then
\[
\frac{\partial}{\partial z}=0,
\]
and 
\[
w=0,
\]
where $\uu = (u,v,w)$.
Cauchy's equation
\[
\sigma_{xx,x} + \sigma_{xy,y} =-F_x
\] 
\[
\sigma_{xy,x} + \sigma_{yy,y} =-F_y
\] 

\[
e_{zz} = 0 \Rightarrow \sigma_{zz} = \nu(\sigma_{xx} + \sigma_{yy})
\]

St. Venant compatibility equations are 
\[
e_{xx,yy} + e_{yy,xx} - 2e_{xy,xy}=0
\]
\[
e_{xx} = \frac{1+\nu}{E} (\sigma_{xx}-\nu(\sigma_{xx} + \sigma_{yy} ))
\]
\[
e_{yy} = \frac{1+\nu}{E} (\sigma_{yy}-\nu(\sigma_{xx} + \sigma_{yy} ))
\]
\[
e_{xy} = \frac{1+\nu}{E}\sigma_{xy}
\]
Plug in,
\[
(1-\nu)\sigma_{xx,yy} - \nu\sigma_{yy,yy} + (1-\nu)\sigma_{yy,xx}-\nu\sigma_{xx,xx} = 2\sigma_{xy,xy}
\]
From Cauchy's equations,
\[
2\sigma_{xy,xy} = -\nabla\cdot\FF - \sigma_{xx,xx}-\sigma_{yy,yy}.
\]
\[
\sigma_{xx,yy} + \sigma_{yy,yy}  + \sigma_{yy,xx}  + \sigma_{xx,xx} = -\frac{\nabla\cdot\FF}{1-\nu} 
\]
Use notation
\[
\nabla_H^2 = \frac{\partial^2}{\partial x^2} + \frac{\partial^2}{\partial y^2}
\]
\[
\nabla_H^2 (\sigma_{xx}+\sigma_{yy})= -\frac{\nabla\cdot\FF}{1-\nu} 
\]

\subsection{Plane stress}
Consider a very thin object (plate) to which only in plane forces are applied.
Unsupportd faces, on faces $z=+-h$,
\[
\sigma_{xz} = \sigma_{yz} = \sigma_{zz}=0.
\]
Center surface (at $z=0$) has no $w$ displacement. Characteristic stress in $x, y$ plane is $\Sigma$.
\[
\sigma_{xx}, \sigma_{xy}, \sigma_{xz}\approx \Sigma 
\]
\[
\sigma_{xx,x} +  \sigma_{xy,y}+  \sigma_{xz,z} = - \FF 
\]
\[
\sigma_{xz,z} \approx \frac{\sigma_{xz}}{h},\quad  \sigma_{xz}\approx \frac h L \Sigma
\]
\[
\sigma_{xx,x} \approx \frac{\Sigma}{h}
\]
\[
u = u^0 (x,y) + z u^1 (x,y) + z^2 u^2(x,y)
\]
\[
w = zw^2(x,y)
\]
\[
\te = \te^0 + z\te^1+\cdots
\]
where 
\[
\te^0 = \mx{ u^0_{,x} & \frac1  2 (u^0_{,y} + v^0_{,x}) & \frac 1 2 u^1\\
& v^0_{,y} & \frac 1 2 v^1\\
& & w^1
}
\]













\end{document}


