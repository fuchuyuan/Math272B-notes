%
\documentclass[12pt]{article}
% geometry
\usepackage[left=1in,top=1in,right=1in,bottom=1in,letterpaper]{geometry}
\usepackage{listings}
%\usepackage{algorithm,algorithmic}
\usepackage{amssymb,amsmath}
\usepackage{url}
\usepackage[lined,ruled,linesnumbered]{algorithm2e}
\lstset{language=Python}          % Set your language (you can change the language for each code-block optionally)

\usepackage[usenames]{color}
\usepackage[color,final]{showkeys}
\definecolor{refkey}{gray}{0.8}
\definecolor{labelkey}{gray}{0.8}

\usepackage[normalem]{ulem}
\newcommand{\remove}[1]{{\color{Gray}\sout{#1}}}
\newcommand{\comm}[1]{{\color{red}(#1)}}
\newcommand{\revise}[1]{{\color{blue}#1}}


\usepackage[utf8]{inputenc}

% Default fixed font does not support bold face
\DeclareFixedFont{\ttb}{T1}{txtt}{bx}{n}{12} % for bold
\DeclareFixedFont{\ttm}{T1}{txtt}{m}{n}{12}  % for normal

% Custom colors
\usepackage{color}
\definecolor{deepblue}{rgb}{0,0,0.5}
\definecolor{deepred}{rgb}{0.6,0,0}
\definecolor{deepgreen}{rgb}{0,0.5,0}


\newcommand{\xx}{{\bf x}}  
\newcommand{\bb}{{\bf b}} 
\newcommand{\cc}{{\mathbb C}} 
\newcommand{\uu}{{\bf u}} 
\newcommand{\FF}{{\bf F}} 


\newcommand{\RR}{{\mathbb R}}  
\newcommand{\EE}{{\mathbb E}}  

\newcommand{\Var}{\text{Var}}  
\newcommand{\std}{\text{std}}  

\newcommand{\lambdat}{{\bf \lambda}}
\newcommand{\mx}[1]{\begin{pmatrix}#1\end{pmatrix}}


\begin{document}

\title{Math 272A Spring 2017}
\author{Chuyuan Fu}
\date{}

\maketitle

\section{What is a continuum?}

The quantities that we are interested in, for example displacement, velocity, pressure, stress... can be written as fields.
For example, we can define a displacement field $\uu(\xx, t) : \RR^3\times \RR \rightarrow \RR^3$, which tells us how the matter in a body is displaced when forces are applied. At each point $(\xx,t)$, $\uu$ gives the average displacement of all molecules contained in a small box that is centered at $\xx$.

Specifically, we assume the existence of a length scale $l$ such that if we average over the displacements of all atoms in a box of size $l$, we get a well-defined average.

Imagine we are interested in a quantity $f$. Each molecule has its own value of $f_i$. Lets assume that the $f_i$'s are fluactuating but have the same mean
\[
f_i = \bar f + f_i'
\]
where $\EE f_i'$=0.
Assume Poisson fluctuations $\std(f_i') = \bar f$.
Then our estimator of $\bar f $ is 
\[
\hat f  = \frac 1 N \sum_i f_i
\]
\begin{align*}
\Var (\hat f)  &=  \Var(\sum_i  \frac 1 N f_i' )\\
		&= \frac {1}{ N^2} \sum_i \Var f_i' \\
		&= \frac {1}{ N^2} N \bar f =\frac{\bar f}{N}
\end{align*}
\[
\std(\hat f ) = \frac{\bar f }{\sqrt{N}}.
\]
Relative error is
\[ 
\frac{\std(\hat f )}{\bar f} = \frac{1}{\sqrt N}
\]
Practically, if I want 0.001 relative error, $N = 10^6$, so I want 100 molecules per side of the box, i.e. need $l \approx 100\times d  = 10$ nm, where d is a molecule diameter, $d = 10^{-10}$m. 
Many of the objects we want to study are larger than this, e.g. in a microfluid device, channel diameter is $10\times 10^{-6} $ m, cell diameter is $10^{-6}$ - $10^{-5}$ m, cell membrane thickness is 10 - 30 nm, carbon nanotube diameter 2 - 200 nm.

In all of the calculus arguments that follow, we will be taking limits, e.g taking a small volume, $\delta V$ and letting $\delta V \rightarrow 0$. In fact, we assume  $\delta V \rightarrow 0$ but $\delta V > l^3$.

\section{Principles of Linear Elasticity}
Assume we have a body to which a discrete set of estimated forces $\{\FF_i\}$ is applied under the constraints
\begin{align*}
&\sum_i \FF_i = 0 \\
&\sum_i \xx_i\times \FF_i = 0 \text{ ( independent of choice of origin)}
\end{align*}

Hooke's 3 laws for linear elasticity
\begin{itemize}
\item If no forces are applied to the body ($\FF_i = 0$), the body has a unique rest state.
we can measure displacements relative to this state.
\item If we are interested in the displacement $\uu$ at a particular point $\uu(P)$ ther exists a set of coeffcicient matrices $\cc_i(P)$ such that  $\uu(P) = \sum_i \cc_i(P) \cdot \FF_i$
\end{itemize}









\end{document}


